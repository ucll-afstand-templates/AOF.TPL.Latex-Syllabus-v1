\chapter{Huisstijl}

\section{Iconen}

\bijkomendMateriaal{Bijkomend materiaal of referentie, gebruik \\ \code{\textbackslash bijkomendMateriaal\{text-to-display\}}}
\bijkomendeInfo{Bijkomende informatie, gebruik \\ \code{\textbackslash bijkomendeInfo\{text-to-display\}}}
\doelstelling{Doelstelling, gebruik \\ \code{\textbackslash doelstelling\{text-to-display\}}}
\oefening{Oefening, taak of opdracht voor de lezer, gebruik \\ \code{\textbackslash oefening\{text-to-display\}}}
\tip{Tip voor de gebruiker, gebruik \\ \code{\textbackslash tip\{text-to-display\}}}
\verdieping{Verdieping, gebruik \\ \code{\textbackslash verdieping\{text-to-display\}}}
\vraag{Vraag aan de lezer, gebruik \\ \code{\textbackslash vraag\{text-to-display\}}}
\waarschuwing{Waarschuwing, gebruik\\ \code{\textbackslash waarschuwing\{text-to-display\}}}

\section{Code in tekst}
Gebruik \code{\textbackslash code\{int a=3\}} om code in de tekst te tonen  (bijvoorbeeld \code{int a = 3}).

Voor langere stukken gebruik je \code{lstlisting}.
\begin{lstlisting}[caption={My label}, label={lst:label}]

public class UserService { 

// bewaar voorlopig alle gebruikers in een lijst. Later zullen we connecteren met een databank 

    private List<User> userRepo = new ArrayList<>(); 
 \end{lstlisting}



